\documentclass{../juni}

\author{Juni Kim}
\title{Lecture 1 Handout}
\date{January 20, 2026}

\begin{document}

\maketitle

\tableofcontents

\newpage

\section{Shell Script Execution Basics}

This section covers execution details that are easy to miss in lecture but
cause many confusing failures.

\subsection{Shebangs}

A shell script is executed by the kernel, not by your shell. The first line
(the \textbf{shebang}) determines which interpreter is used.

\begin{lstlisting}[language=bash]
#!/usr/bin/env bash
\end{lstlisting}

Important points:
\begin{itemize}
  \item The kernel reads the shebang and invokes the interpreter.
  \item \texttt{/usr/bin/env bash} is preferred over \texttt{/bin/bash}.
  \item \texttt{/bin/bash} is not guaranteed to exist on all systems.
  \item \texttt{/bin/sh} is the only interpreter required by POSIX.
\end{itemize}

If the shebang is missing or invalid, execution will fail even if the file
exists.

\subsection{Executable Bit}

A script must be marked executable to be run directly:

\begin{lstlisting}[language=bash]
chmod +x script.sh
./script.sh
\end{lstlisting}

Contrast with:

\begin{lstlisting}[language=bash]
bash script.sh
\end{lstlisting}

Running with \texttt{bash script.sh} bypasses the shebang entirely. If a script
only works this way, it may fail in grading or CI.

\subsection{\texttt{\$0}, \texttt{dirname}, and \texttt{realpath}}

Inside a script, \texttt{\$0} expands to how the script was invoked, not
necessarily its absolute path.

Common defensive pattern:

\begin{lstlisting}[language=bash]
DIR="$(realpath "$(dirname "$0")")"
\end{lstlisting}

This allows scripts to:
\begin{itemize}
  \item locate files relative to themselves
  \item be run from any working directory
  \item avoid reliance on caller state
\end{itemize}

\newpage
\section{Variables and Quoting}

This section covers variable behavior and quoting rules that frequently cause
subtle bugs.

\subsection{Shell Variables vs Environment Variables}

Shell variables are local to the shell. Environment variables are inherited by
child processes.

\begin{lstlisting}[language=bash]
FOO=hello        # shell variable
export BAR=world # environment variable
\end{lstlisting}

Only exported variables are visible to programs you run.

\subsection{Capturing Command Output}

Use command substitution to capture stdout:

\begin{lstlisting}[language=bash]
FILES="$(ls)"
COUNT="$(ls | wc -l)"
\end{lstlisting}

Exit codes are \textbf{not} captured—only stdout.

\subsection{Quoting and Word Splitting}

Unquoted variables are split on whitespace.

\begin{lstlisting}[language=bash]
FILE="my file.txt"
rm $FILE      # WRONG: expands to two arguments
rm "$FILE"    # correct
\end{lstlisting}

Rule of thumb:
\begin{quote}
Quote variables unless you explicitly want splitting.
\end{quote}

\subsection{Globbing}

Globs (\texttt{*}, \texttt{?}, \texttt{[...]}) expand \textbf{before} program
execution.

\begin{lstlisting}[language=bash]
echo *.txt
\end{lstlisting}

If no files match, the glob may remain unexpanded, depending on shell settings.

\newpage
\section{Shell Misc I: Control Flow Footguns}

This section covers control-flow behaviors that are often misunderstood.

\subsection{Exit Codes}

Every command returns an exit code:
\begin{itemize}
  \item \texttt{0} = success
  \item non-zero = failure
\end{itemize}

The shell variable \texttt{\$?} contains the last exit code.

\subsection{\texttt{\&\&} and \texttt{||}}

\begin{lstlisting}[language=bash]
cmd1 && cmd2   # cmd2 runs only if cmd1 succeeds
cmd1 || cmd2   # cmd2 runs only if cmd1 fails
\end{lstlisting}

These operators are driven entirely by exit codes.

\subsection{\texttt{test}, \texttt{true}, and \texttt{false}}

\begin{lstlisting}[language=bash]
test -f file.txt
[ -d somedir ]
true
false
\end{lstlisting}

\texttt{test} and \texttt{[ ]} are just commands that return exit codes.

Negation inverts exit status:

\begin{lstlisting}[language=bash]
! test -e file.txt
\end{lstlisting}

\newpage
\section{Shell Misc II: Defensive Defaults}

This section covers defaults and patterns expected in this course.

\subsection{\texttt{set -euo pipefail}}

Recommended at the top of scripts:

\begin{lstlisting}[language=bash]
set -euo pipefail
\end{lstlisting}

Effects:
\begin{itemize}
  \item \texttt{-e}: exit on unhandled error
  \item \texttt{-u}: error on use of unset variables
  \item \texttt{pipefail}: pipelines fail if any command fails
\end{itemize}

Caveats:
\begin{itemize}
  \item \texttt{-e} does not trigger inside all conditionals
  \item control-flow semantics change subtly
  \item this is not a substitute for careful reasoning
\end{itemize}

\subsection{Idempotent Filesystem Operations}

Filesystem state persists across runs. Scripts must tolerate re-execution.

Prefer:

\begin{lstlisting}[language=bash]
mkdir -p output
rm -f temp.txt
\end{lstlisting}

Be deliberate about:
\begin{itemize}
  \item \texttt{>} vs \texttt{>>}
  \item overwriting vs appending
  \item cleanup on failure
\end{itemize}

Rule:
\begin{quote}
If you did not create it in the script, do not assume it exists.
\end{quote}

\newpage
\section{Docker}

Docker is used to standardize the \textbf{host system}. It does not eliminate
the host/target distinction.

\subsection{Docker Images}

An image is a filesystem snapshot plus metadata.

Key ideas:
\begin{itemize}
  \item Images are built from Dockerfiles.
  \item Images are immutable once built.
  \item Building an image does not run your program.
\end{itemize}

\begin{lstlisting}[language=bash]
docker build -t myimage .
\end{lstlisting}

\subsection{Docker Containers and Volumes}

A container is a running instance of an image.

\begin{lstlisting}[language=bash]
docker run -it myimage
\end{lstlisting}

Volumes and bind mounts allow sharing directories with the host:

\begin{lstlisting}[language=bash]
docker run -it -v "$PWD:/work" myimage
\end{lstlisting}

Important:
\begin{itemize}
  \item Mounted directories are shared mutable state.
  \item Files created in the container may persist on the host.
  \item Deleting a container does not delete volumes.
\end{itemize}

For debugging:
\begin{itemize}
  \item Containers can be stopped and restarted.
  \item Images must be rebuilt to change their contents.
\end{itemize}

\subsection{Inspecting and Managing Containers}

Docker keeps track of containers independently of your shell. A very common
source of confusion is forgetting which containers are running.

\subsubsection{\texttt{docker ps}}

The command \texttt{docker ps} lists \textbf{currently running containers}:

\begin{lstlisting}[language=bash]
$ docker ps
CONTAINER ID   IMAGE        COMMAND                  CREATED         STATUS         PORTS     NAMES
a3c1f4b92d3e   debian:12    "/bin/bash"              3 minutes ago   Up 3 minutes             hopeful_morse
\end{lstlisting}

Important columns:
\begin{itemize}
  \item \texttt{CONTAINER ID}: unique identifier (shortened hash)
  \item \texttt{IMAGE}: image the container was created from
  \item \texttt{COMMAND}: entrypoint / command being run
  \item \texttt{STATUS}: whether the container is running
  \item \texttt{NAMES}: human-readable name (auto-generated if not specified)
\end{itemize}

To see \textbf{all containers}, including stopped ones:

\begin{lstlisting}[language=bash]
docker ps -a
\end{lstlisting}

Stopped containers still exist and still consume metadata until removed.

\subsubsection{Stopping Containers}

To stop a running container:

\begin{lstlisting}[language=bash]
docker stop hopeful_morse
\end{lstlisting}

You may also use the container ID:

\begin{lstlisting}[language=bash]
docker stop a3c1f4b92d3e
\end{lstlisting}

Stopping a container sends a termination signal to the process inside.

\subsubsection{Removing Containers}

To permanently delete a container:

\begin{lstlisting}[language=bash]
docker rm hopeful_morse
\end{lstlisting}

Only stopped containers can be removed. To stop and remove in one step:

\begin{lstlisting}[language=bash]
docker rm -f hopeful_morse
\end{lstlisting}

This is often useful during development.

\subsubsection{Common Pitfalls}

\begin{itemize}
  \item Exiting a shell inside a container does not delete the container.
  \item Containers can accumulate if not explicitly removed.
  \item Volumes and bind mounts persist even after containers are deleted.
\end{itemize}

Rule of thumb:
\begin{quote}
If something strange is happening, run \texttt{docker ps -a} first.
\end{quote}

\end{document}
