\documentclass{../lecture}

\title{Lecture 5}
\date{January 29, 2026}

\begin{document}

\maketitle

\begin{frame}{Today's Lecture}
  \begin{itemize}
    \item No essential new material, bit more relaxed pace for questions and lab
      hours
    \item Will be reviewing concepts from the last lectures
    \item Some other parts of a Linux System that you might use
  \end{itemize}
\end{frame}

\begin{frame}{Writing Good Shell Programs}
  \begin{itemize}
    \item We want to write programs that behave predictably and are defensive to
      \begin{itemize}
        \item Working directory
        \item Environment variables
        \item Filesystem contents
      \end{itemize}
    \item Your program must behave predictably in hostile filesystem
      environments or error cleanly.
  \end{itemize}
\end{frame}

\begin{frame}{Critical Distinctions}
  \begin{itemize}
    \item Host vs Target: We are building up a bootable image for a target
      system on a host machine. (Different characteristics)
    \item Source vs Artifact: Developers give us source code, the compiler emits
      artifacts. Out-of-tree builds separate these.
    \item Initramfs vs Root filesystem: Initramfs is temporary, it boots the
      root filesystem and runs \texttt{switch\_root}.
  \end{itemize}
\end{frame}

\begin{frame}{The C Compiler and Friends}
  \begin{itemize}
    \item Preprocessor, Compiler, Linker stages
    \item Corresponding flags are \texttt{CPPFLAGS}, \texttt{CFLAGS}, \texttt{LDFLAGS}
    \item You need to ensure the host libc does not contaminate your build.
    \item \texttt{./configure} $\rightarrow$ \texttt{make} $\rightarrow$
      \texttt{make install} (Roughly, differs for different source)
    \item Please parallelize make with the \texttt{-j} option!
  \end{itemize}
\end{frame}

\begin{frame}{Building the Kernel}
  \begin{itemize}
    \item You can configure the kernel with \texttt{make menuconfig}
    \item See handout for drivers you should not disable (you might not have a
      console or networking without these)
    \item Booting as an EFI Stub
    \item Configure default command line (Firmware won't give you anything)
    \item Embed initramfs into the kernel (you don't have a filesystem when
      booting as an EFI stub)
  \end{itemize}
\end{frame}

\begin{frame}{User Daemons and System Configuration}
  \begin{itemize}
    \item Most configuration files will go in \texttt{/etc}, including declaring
      users.
    \item Busybox init reads \texttt{/etc/inittab} to see what it should start
      when.
  \end{itemize}
  \begin{block}{Important Daemons init launches}
    \begin{itemize}
      \item Getty - login screen
      \item Dhcpcd - IP address configuration + internet
      \item Chrony - Time
      \item Eudev - device discovery
    \end{itemize}
  \end{block}
\end{frame}

\begin{frame}{Disk Partitions and Filesystems}
  \begin{itemize}
    \item \texttt{mkfs} initializes filesystems
    \item \texttt{sgdisk} helps you assemble disk partitions together.
    \item We have two partitions:
      \begin{enumerate}
        \item ESP - readable by firmware (\texttt{/dev/vda1})
        \item Ext4 - actually useful as a root filesystem (\texttt{/dev/vda2})
      \end{enumerate}
    \item GPT header and metadata tells firmware where the ESP is, the firmware
      will locate the kernel inside the ESP.
  \end{itemize}
\end{frame}

\begin{frame}{Some Debugging Pointers}
  \begin{itemize}
    \item No output: serial console or kernel cmdline issue.
    \item Panic early: initramfs issue or missing /init.
    \item Boots but no login: getty, /etc/inittab, or permissions.
    \item Networking dead: dhcpcd user, device nodes, or kernel config.
  \end{itemize}
\end{frame}

\section{Other things you might want to think about}

\begin{frame}{OpenSSL}
  \begin{itemize}
    \item Major library not covered in this assignment.
    \item Implements cryptographic protocols.
    \item Used for SSH, HTTPS, \dots
  \end{itemize}
\end{frame}

\begin{frame}{SSH}
  \begin{itemize}
    \item Protocol used for securely accessing servers
    \item You probably accessed most linux servers via ssh
    \item sshd is responsible for this
  \end{itemize}
\end{frame}

\begin{frame}{SUID Binaries}
  \begin{itemize}
    \item Normally, programs run with the UID/GID of the invoking user.
    \item Some binaries are marked \textbf{SUID} (set-user-ID).
    \item When executed, they run with the \emph{file owner's} UID instead.
    \item Common use: controlled privilege escalation (e.g. \texttt{sudo},
      \texttt{doas}, \texttt{passwd}).
    \item Optional: try configuring doas on your operating system!
  \end{itemize}
\end{frame}

\begin{frame}{Timezones}
  \begin{itemize}
    \item Linux systems use unix millis (time since Jan 1, 1970).
    \item Timezones are the responsibility of userspace
    \item Timezone info lives under \texttt{/usr/share/zoneinfo}.
    \item You generally symlink one of the zoneinfo files to \texttt{/etc/localtime}
  \end{itemize}
\end{frame}

\begin{frame}{Filesystem Mounts and \texttt{/etc/fstab}}
  \begin{itemize}
    \item \texttt{/etc/fstab} declares what should be mounted and where (instead
      of editing init which can be catastrophic).
    \item Entries describe:
      \begin{itemize}
        \item device or filesystem
        \item mount point
        \item filesystem type
      \end{itemize}
    \item \texttt{mount -a} will consult fstab.
  \end{itemize}
\end{frame}

\begin{frame}{System Hostname}
  \begin{itemize}
    \item The kernel maintains a system-wide hostname (default set in kernel config).
    \item Exposed via \texttt{/proc/sys/kernel/hostname}.
    \item You can write to this file to change the hostname on statup.
    \item Used by shells, networking tools, and system services.
  \end{itemize}
\end{frame}

\begin{frame}{GUI on Linux}
  \begin{itemize}
    \item X11 display protocol (+ x.org server)
    \item Window managers on top of the x.org server
    \item Wayland (replacing xorg on desktop)
  \end{itemize}
\end{frame}

\begin{frame}{Up Next}
  \begin{itemize}
    \item Final Lab Hours
    \item Submit the lab by Friday 11:59PM
    \item Thank you for participating!
  \end{itemize}
\end{frame}

\end{document}
