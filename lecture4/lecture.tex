\documentclass{../lecture}

\title{Lecture 4}
\date{January 26, 2026}

\begin{document}

\maketitle

% make sure you cover boot levels, especially in init.

\begin{frame}{From Previous Lectures}
  \begin{itemize}
    \item Build shell scripts with predictable behavior in spite of:
      \begin{itemize}
        \item Working Directory
        \item Undeclared environment variables
        \item State of the filesystem
      \end{itemize}
    \item The kernel unpacks an initramfs, which is purely in memory. We mount
      our disk, then \texttt{switch\_root}.
  \end{itemize}
\end{frame}

\begin{frame}{This Lecture}
  \begin{itemize}
    \item We close the story on userspace and filesystems.
    \item Part 1: User programs, root filesystem, and configuration
    \item Part 2: Partitions and Filesystems
  \end{itemize}
\end{frame}

\begin{frame}{The FHS (Again)}
  The root filesystem needs to follow the filesystem hierarchy standard even
  more strictly than the rootfs.
  \begin{block}{You need these:}
    \begin{itemize}
      \item \texttt{/usr/bin}, \texttt{/usr/share}, \texttt{/usr/include}
      \item \texttt{/lib}, \texttt{/bin}
      \item \texttt{/etc}
      \item \texttt{/run}, \texttt{/proc}, \texttt{/sys}, \texttt{/dev}
    \end{itemize}
  \end{block}
  In effect, a near-superset of your initramfs tree.
\end{frame}

\begin{frame}{PID 1}
  Usually, PID 1 will:
  \begin{itemize}
    \item Reap orphaned zombies (essential!)
    \item Mounts filesystems
    \item Spin up daemons
    \item Maybe other functionality too
  \end{itemize}
  \begin{exampleblock}{Examples of Init}
    \begin{itemize}
      \item Busybox init (we will use this for simplicity)
      \item Systemd (popular but controversial!)
      \item OpenRC, runit
    \end{itemize}
  \end{exampleblock}
\end{frame}

\begin{frame}{Configuring PID 1}
  \begin{itemize}
    \item \texttt{/etc/inittab} describes \emph{what} should run and \emph{when}
    \item Entries can be one-shot, respawned, or tied to boot stages (startup,
      shutdown, \dots)
    \item Long-running services must be restarted if they die
  \end{itemize}
  \begin{exampleblock}{Startup scripts}
    \begin{itemize}
      \item Typically a script like \texttt{/etc/init.d/rcS}
      \item Responsible for mounting filesystems and starting daemons
      \item Invoked automatically by \texttt{init}
    \end{itemize}
  \end{exampleblock}
\end{frame}

\begin{frame}{Configuring Users and Groups}
  \begin{itemize}
    \item The most basic way in which the operating system regulates access to
      privileged resources.
    \item You only need to have \texttt{root} and \texttt{dhcpcd} (see handout
      for more info on dhcpcd)
    \item Root is all-powerful, generally avoid being root because you can break
      a lot of things!
  \end{itemize}
  \begin{block}{Configuration files}
    \begin{itemize}
      \item \texttt{/etc/passwd}: Contains metadata about users
      \item \texttt{/etc/shadow}: Contains hashed passwords (needs restrictive permissions)
      \item \texttt{/etc/group}: Contains metadata about groups
    \end{itemize}
  \end{block}
\end{frame}

\section{Important userspace daemons}

\begin{frame}{Dhcpcd}
  \begin{itemize}
    \item You basically need DHCP on modern networks for any kind of internet
      access
    \item The kernel will not do IP address configuration for you
    \item You can test working by \texttt{ping 1.1.1.1} (or some other raw ip
      address; DNS will be covered later)
    \item Dhcpcd requires its own user and group, along with a writable
      directory of its own! (See handout)
  \end{itemize}
\end{frame}

\begin{frame}{Other Essential Daemons (You must start them!)}
  \begin{block}{getty}
    \begin{itemize}
      \item Spawns login prompts on \texttt{/dev/tty*}
      \item One instance per terminal
    \end{itemize}
  \end{block}
  \begin{block}{eudev}
    \begin{itemize}
      \item Userspace device manager
      \item Creates device nodes under \texttt{/dev}
      \item Triggers driver loading and hotplug events
    \end{itemize}
  \end{block}
  \begin{block}{chrony}
    \begin{itemize}
      \item NTP client for wall-clock time
      \item Corrects clock drift after boot
      \item Required for TLS, sane logs, reproducible builds (assumes monotonic
        time)
    \end{itemize}
  \end{block}
\end{frame}

\section{Filesystems}

\begin{frame}{What is a Filesystem?}
  \begin{itemize}
    \item A filesystem defines how bytes on disk are interpreted
    \item The filesystem interface is in the kernel (Linux is monolithic!)
    \item It maps names to data and metadata
    \item It defines persistence across reboots
  \end{itemize}
  \begin{block}{Responsibilities}
    \begin{itemize}
      \item File and directory naming
      \item Permissions and ownership
      \item Metadata (timestamps, size, links)
      \item Crash recovery and consistency guarantees
    \end{itemize}
  \end{block}
\end{frame}

\begin{frame}{Filesystems we use}
  \begin{block}{Ext4}
    \begin{itemize}
      \item Standard Linux filesystem
      \item Journaling (crash recovery)
      \item Allows for permissions
    \end{itemize}
  \end{block}
  \begin{block}{VFat}
    \begin{itemize}
      \item Readable by EFI Firmware
      \item Much simpler, no permissions or even symlinks!
      \item We will put just the kernel in a special path.
    \end{itemize}
  \end{block}
\end{frame}

\begin{frame}{GPT}
  \begin{itemize}
    \item GUID Partition Table (replaces MBR)
    \item A standardized partition table format
    \item Unique identifiers for each partition
    \item Well-defined partition types (not just numbers)
  \end{itemize}
  \begin{block}{Regarding Firmware}
    \begin{itemize}
      \item Firmware reads metadata at the front of the disk
      \item Special partition types signal boot-related data
      \item The EFI System Partition (ESP) is identified this way
    \end{itemize}
  \end{block}
  You should consult the handout for further info!
\end{frame}

\begin{frame}{Before we keep moving}
  There are some toolchain constraints in the assignment!
  \begin{itemize}
    \item We will use filesystem debug tools and \texttt{sgdisk}
    \item LFS manuals will often get you to create a loopback device, which is
      impossible inside a docker container.
    \item If you attempt to copy files into filesystems via loopback devices,
      your build pipeline will fail.
  \end{itemize}
\end{frame}

\begin{frame}{Boot Path Summarized}
  (Cue blackboard)
  \begin{enumerate}
    \item Firmware starts
    \item Firmware figures out where ESP is from GPT metadata
    \item Launch bootloader (aka kernel) inside ESP (fixed path)
    \item Kernel unpacks initramfs
    \item initramfs \texttt{/init} mounts the real filesystem, then
      runs \texttt{switch\_root}
    \item \texttt{/sbin/init} in the disk takes over and launches programs.
    \item login screen (hopefully)
  \end{enumerate}
\end{frame}

\begin{frame}{Up Next}
  \begin{itemize}
    \item Lab hours on Tuesday 1-5PM!
    \item Thursday's lecture will not feature topics critical to the completion
      of your assignment!
    \item Some straggling topics: DNS
    \item Topics beyond the lab: Graphical Environments, Dbus, Certificates, \dots
    \item Course Wrap-up, final lab hours
  \end{itemize}
\end{frame}

\end{document}
