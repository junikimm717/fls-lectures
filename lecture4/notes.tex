\documentclass{../juni}

\author{Juni Kim}
\title{Lecture 4 Handout}
\date{January 26, 2026}

\begin{document}

\maketitle

\tableofcontents

\newpage

\section{Userspace Configuration}

At this point in the course, the kernel is capable of booting and executing an
\texttt{initramfs}. This handout focuses on configuring the \emph{persistent}
userspace environment that lives on disk and takes over after
\texttt{switch\_root}.

Everything in this section assumes that:
\begin{itemize}
  \item the kernel has successfully booted
  \item \texttt{/sbin/init} is executed as PID 1
  \item BusyBox is providing the init implementation
\end{itemize}

\subsection{(Important) Information Regarding Dhcpcd}

\texttt{dhcpcd} is a DHCP client responsible for acquiring an IP address on
modern networks. The kernel does \emph{not} configure networking automatically;
without a userspace DHCP client, your system will have no internet access.

Unlike many simple utilities, \texttt{dhcpcd} assumes the existence of:
\begin{itemize}
  \item a dedicated \texttt{dhcpcd} user
  \item a dedicated \texttt{dhcpcd} group
  \item a writable runtime directory under \texttt{/var/run}
\end{itemize}

In particular, the directory:

\begin{verbatim}
/var/run/dhcpcd
\end{verbatim}

\noindent must:
\begin{itemize}
  \item exist in the root filesystem image
  \item be owned by \texttt{dhcpcd:dhcpcd}
  \item be writable by that user
\end{itemize}

Because your root filesystem image is constructed offline, these ownership and
permission adjustments must be made using filesystem debugging tools rather
than normal runtime commands. Refer to the \textbf{Disk Tools} section below for
details on how to use \texttt{debugfs} to accomplish this.

If these permissions are incorrect, \texttt{dhcpcd} will fail silently or exit
early during boot.

\subsection{Configuring user and group files}

Linux systems describe users and groups using plain text configuration files
under \texttt{/etc}. These files are read by libc and userspace programs; the
kernel itself does not interpret them.

You must configure the following files:

\begin{itemize}
  \item \texttt{/etc/passwd}
  \item \texttt{/etc/shadow}
  \item \texttt{/etc/group}
\end{itemize}

\texttt{/etc/passwd} contains basic user metadata:
\begin{verbatim}
username:x:uid:gid:comment:home:shell
\end{verbatim}

\texttt{/etc/group} contains group metadata:
\begin{verbatim}
groupname:x:gid:user1,user2
\end{verbatim}

\texttt{/etc/shadow} stores password hashes and must be readable only by root:
\begin{verbatim}
username:hashed-password:...
\end{verbatim}

For this course, you minimally need:
\begin{itemize}
  \item a \texttt{root} user (UID 0)
  \item a \texttt{dhcpcd} user and group
\end{itemize}

The permissions on \texttt{/etc/shadow} are \emph{security critical}. If this file
is world-readable, your system is misconfigured. Because these files live inside
an ext4 image, permissions must be verified and corrected using \texttt{debugfs}
after filesystem creation.

\subsection{Busybox inittab}

BusyBox \texttt{init} determines system behavior by reading
\texttt{/etc/inittab}. This file describes \emph{what} programs should be run
and \emph{under what conditions}.

An \texttt{inittab} entry conceptually consists of:
\begin{verbatim}
<id>:<runlevels>:<action>:<command>
\end{verbatim}

You will encounter entries equivalent in structure to the following:

\begin{verbatim}
::sysinit:<startup-script>
console::respawn:<login-program>
::ctrlaltdel:<reboot-command>
::shutdown:<cleanup-command>
\end{verbatim}

Important ideas:
\begin{itemize}
  \item \texttt{sysinit} entries run very early, once
  \item \texttt{respawn} entries are restarted if they exit
  \item \texttt{init} is responsible for keeping essential services alive
\end{itemize}

Paths and arguments in \texttt{inittab} are \emph{policy}. If you copy them
without understanding:
\begin{itemize}
  \item what program is being executed
  \item why it needs to persist
  \item what happens if it exits
\end{itemize}
then you will have a system that is difficult to debug.

\subsection{Startup script}

Rather than placing all logic directly into \texttt{inittab}, it is customary to
delegate early boot tasks to a startup script.

This script is invoked by \texttt{init} during the \texttt{sysinit} stage and is
responsible for:
\begin{itemize}
  \item mounting kernel-provided filesystems
  \item preparing runtime directories
  \item launching essential daemons
\end{itemize}

A conceptual outline of such a script might look like:

\begin{lstlisting}[language=bash]
#!/bin/sh

# mount kernel interfaces
mount kernel-exposed-filesystem A
mount kernel-exposed-filesystem B

# prepare runtime state
create volatile directories

# start long-running services
start service X
start service Y
\end{lstlisting}

The exact programs and mount options matter less than understanding that:
\begin{itemize}
  \item this script runs as PID 1 initially
  \item failure here usually prevents login entirely
  \item ordering is significant
\end{itemize}

\subsection{Nameserver Configuration (Optional)}

Name resolution is configured via \texttt{/etc/resolv.conf}. This file tells
userspace programs where to send DNS queries.

A minimal example might specify one or more nameservers:
\begin{verbatim}
nameserver 1.1.1.1
nameserver 8.8.8.8
\end{verbatim}

With a valid \texttt{resolv.conf}, programs that rely on DNS (such as
\texttt{wget}) can resolve hostnames.

Note that this is \emph{not sufficient} for HTTPS. TLS requires:
\begin{itemize}
  \item cryptographic libraries
  \item a trusted certificate store
\end{itemize}

Those topics are intentionally deferred to later lectures.

\newpage

\section{Disk Tools}

This section describes the tools used to construct and inspect disk images.
These tools operate on \emph{files}, not block devices, due to container
constraints.

\subsection{Generic}

\texttt{dd} is a low-level byte copying tool. In this course, it is primarily
used to create zero-initialized disk images.

Conceptually:
\begin{itemize}
  \item the output file represents a disk
  \item the size determines the disk capacity
\end{itemize}

\texttt{sgdisk} is used to create and modify GPT partition tables. It operates
directly on disk images and allows you to:
\begin{itemize}
  \item define partition boundaries
  \item assign partition types
  \item label partitions
\end{itemize}

Because Docker does not permit loopback devices, partitions must be stitched
together manually by copying filesystem images into specific offsets of a larger
disk image. Understanding sector size and offsets is essential; blindly copying
commands without reasoning about these values will lead to corrupted images.

\subsection{Ext4}

\texttt{ext4} is the primary Linux filesystem used for the root filesystem.

Relevant tools include:
\begin{itemize}
  \item \texttt{mkfs.ext4} / \texttt{mke2fs}: create ext4 filesystems
  \item \texttt{debugfs}: inspect and modify ext4 images offline
\end{itemize}

\texttt{debugfs} is particularly important in this course. It allows you to:
\begin{itemize}
  \item adjust file permissions
  \item change ownership (note you must change user and group owners by UID and GID, not name)
  \item fix mistakes without rebuilding everything
\end{itemize}

Its interface is interactive and non-obvious. Typical usage involves opening an
image, issuing commands, and explicitly exiting. This tool is required to:
\begin{itemize}
  \item restrict access to \texttt{/etc/shadow}
  \item configure ownership for \texttt{dhcpcd} runtime directories
\end{itemize}

\subsection{VFat}

The EFI System Partition must use a FAT-based filesystem, typically FAT32.

Tools from \texttt{dosfstools} are used:
\begin{itemize}
  \item \texttt{mkfs.vfat}: create the filesystem
  \item \texttt{mmd}: create directories inside the image
  \item \texttt{mcopy}: copy files into the image
\end{itemize}

EFI firmware does not search arbitrarily. The kernel must reside at a
well-defined path within the ESP, and the exact filename depends on the target
architecture.

This filesystem does not support permissions, symlinks, or Unix ownership.
It exists solely as a firmware-readable transport mechanism.

\end{document}
